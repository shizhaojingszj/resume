% !TEX TS-program = xelatex
% !TEX encoding = UTF-8 Unicode
% !Mode:: "TeX:UTF-8"

\documentclass{resume}
\usepackage{zh_CN-Adobefonts_external} % Simplified Chinese Support using external fonts (./fonts/zh_CN-Adobe/)
%\usepackage{zh_CN-Adobefonts_internal} % Simplified Chinese Support using system fonts
\usepackage{linespacing_fix} % disable extra space before next section
\usepackage{cite}
\usepackage{hyperref}

\begin{document}
\pagenumbering{gobble} % suppress displaying page number

\name{赵孟}

% {E-mail}{mobilephone}{homepage}
% be careful of _ in emaill address
\contactInfo{(+86) 186-xxxx-5048}{shizhaojingszj@163.com}{Web全栈/数据工程师}{GitHub @shizhaojingszj}
% {E-mail}{mobilephone}
% keep the last empty braces!
%\contactInfo{xxx@yuanbin.me}{(+86) 131-221-87xxx}{}
 
\section{个人总结}
前后端web开发及数据分析相关经验,熟练掌握多种编程语言,全栈工程师。熟悉常用软件开发工具,擅长学习GitHub上流行的开源软件技术并应用到工作中,时刻保持对提高软件开发效率,提高流程可重复性的积极性。
有生物信息流程开发经验,图像识别深度学习算法开发及部署经验,多种web本地开发及云开发技术栈,DevOp能力出色,两年小团队管理经验,从未离开编码一线。有科研背景和创新思维,热爱学习新事物。经历小微创业公司成立到game over。

% \section{\faGraduationCap\ 教育背景}
\section{教育背景}
\datedsubsection{\textbf{中国科学院大学/技术生物与农业工程研究所},生物物理,\textit{博士}}{2006.9 - 2014.6}
\datedsubsection{\textbf{中国农业大学},生命科学,\textit{理学学士}}{2002.9 - 2006.6}

% \section{\faCogs\ IT 技能}
\section{技术能力}
% increase linespacing [parsep=0.5ex]
\begin{itemize}[parsep=0.2ex]
  \item \textbf{编程语言}: 熟练应用Python(8年)、R(数据可视化、统计)、Javascript(Typescript、Node.js)、Go、Groovy等编程语言,了解Java、Scala、Clojure等语言。可用\LaTeX{}产生标准化文档(论文、简历等)。
  \item \textbf{操作系统}: Linux(Ubuntu) / ESXi(6.7)
  \item \textbf{数据库}: MySQL(8.0) / MongoDB / AWS RDS(Postgresql 10)
  \item \textbf{后端框架}: SQLAlachemy / Flask / PyTest / Echo(Go) / Express.js / AWS Amplify
  \item \textbf{前端框架}: OpenSeadragon.js / OpenLayers.js / React / Vue2
  \item \textbf{DevOp}: Ansible / Vagrant(Ruby) / Docker Compose / Jsonnet / Nextflow(Groovy) / Jenkins / AWS CDK
  \item \textbf{图像处理}: OpenCV(Python) / OpenSlide(Python) / QuPATH(Groovy)
  \item \textbf{数据分析}: Pandas / Numpy / Scipy / sklearn / tidyverse / ggplot2 / seaborn
  \item \textbf{人工智能}: PyTorch / TensorFlow / Keras
  \item \textbf{爬虫}: Scrapy / requests(Python)
  \item \textbf{其他关键词}: IoT / Jetson / Reproduciblity / GraphQL
  \item \textbf{英语}: 6级 / 可不用字幕听懂YT的技术类视频
\end{itemize}

% \end{itemize}

\section{工作经历}
\datedsubsection{\textbf{嘉兴清格医疗科技有限公司 | SingularityAI}, 技术总监}{2021.12-2022.6}
\begin{itemize}
%   \item 飞猪北京前端团队全面负责各交通线的票务(机票/火车票/汽车票) web 应用与事业群基础架构研发
  \item 负责组织前后端软件开发,分配任务及监督开发进度。
  \item 负责软件开发相关会议的主持,负责主持JIRA和Confluence日常更新,对软件开发进行质量控制。
  \item 监督算法流程的开发过程,负责与算法流程对接(主要是病理图像分类和检测算法)。
    \begin{itemize}
    \item 利用AWS云平台,将数字病理浏览分享功能云端化。使用S3、AppSync、RDS、CDK等。构建基于AWS平台的web应用。前端使用React + TailwindCSS快速开发。
    \item 引入Jsonnet和Nextflow(Groovy)等工具构建更加自动化的算法流程,提高算法流程的结果可重复性。
    \end{itemize}
\end{itemize}

\datedsubsection{\textbf{嘉兴清格医疗科技有限公司 | SingularityAI}, 研发总监}{2020.4-2021.12}
\begin{itemize}
  \item 负责与产品同事对接需求,组织前后端软件开发,分配任务及监督开发进度,主导单元测试编写。
  \item 负责软件开发相关会议的主持,组织撰写开发计划书,软件开发概要设计,详细设计等相关文档,对软件开发进行质量控制。
  \item 完成多项公司内部和外部软件开发任务:
    \begin{itemize}
    \item 一款整合浏览和算法展示的软件,已交付给医院客户。使用微服务的方式,5个组件协同作用,可用于同时部署在CPU服务器和GPU服务器。
    \item 使用QuPATH的脚本功能(Groovy)整合Stardist(一开源细胞核识别算法,JAVA)流程。
    \item 使用TypeScript的oclif框架构建命令行应用,也整合了Express,作为QuPath算法调用的中间组件。
    \item 使用Ansible自动化软件开发部署流程。
    \item 使用Node.js + Express + PM2构建IoT项目的后端。
    \item 使用AWS中的Amplify(AppSync为核心,强调GraphQL)快速开发框架,
    % 但中国区目前无法使用该框架,仅使用Amplify Mock用于开发简易项目。
    结合新版React技术栈,使用React Hooks构建SPA应用。
    \end{itemize}
\end{itemize}

\datedsubsection{\textbf{嘉兴慧领医疗科技有限公司/嘉兴清格医疗科技有限公司 | SingularityAI}, 高级软件工程师}{2018.10-2020.4}
\begin{itemize}
  \item 公司人事变动,在2020.1进行过一次更名
  \item 以主要后端开发者的身份开发一款单机版数字病理软件(Flask、MySQL、Docker),包括病理切片的上传、权限控制(JWT)、数字病理传递流程等。均使用Flask的best practice全家桶。对PyTest测试框架大量应用。
  \item 使用Go构建单文件web应用(Echo框架),实现上述软件“自动更新”功能。学习使用Go的动力是在于其单文件易于部署的特性且语言本身简单易用。
  \item 独立设计和完成AI运行服务(Python、Flask、Docker、MySQL),将各种数字病理相关算法进行容器化部署,支持多任务多GPU同时调用,有资源调度功能。使用MySQL实现的Queue。应用了gRPC技术。
  \item 参与算法流程开发,开发Active Learning流程——一种通过对待标注数据进行筛选而降低标注成本并可能加快现有模型优化过程的方法,对web应用有一定的要求。
  \item 负责算法流程工程化,使用docker-compose对Python-based的算法流程(TensorFlow/PyTorch)进行包装,标准化输入输出,以对接上游的AI运行服务。
  \item 负责对第三方合作商的so文件进行ctypes包装成python可用包,使算法流程支持第三方数据格式。
  \item 尝试用Vue2+Flask构建过一个Images Viewer简单项目,方便同事进行远程阅片。
  \item 以产品经理的身份与第三方外包团队进行需求沟通以及跟进进展,为我司建立在线算法标注平台,方便外部标注人员产生标注数据。
\end{itemize}

\datedsubsection{\textbf{北京雅康博生物科技有限公司 | ACCB}, 生物信息工程师}{2017.9-2018.10}
\begin{itemize}
  \item \textbf{独立负责一个小panel流程开发}将原流程从Perl转成Python,并应用了GoCD等CI技术。
  \item 后跟随直属领导进入创业公司,ACCB创始人参股。
\end{itemize}

\datedsubsection{\textbf{北京创新乐土生物科技有限公司 | Cheerland}, 生物信息工程师}{2016.9-2017.9}
\begin{itemize}
  \item \textbf{独立负责Microbiome流程开发}独立学习并应用了QIIME流程(Python语言包),包括自动化报告产生(Jinja2+HTML)。
\end{itemize}

\datedsubsection{\textbf{北京诺禾致源生物科技有限公司 | Novogene}, 生物信息工程师}{2014.6-2016.9}
\begin{itemize}
  \item \textbf{独立负责eQTL流程开发}使用matrixEQTL构建流程。
  \item RNA项目经验:从二代测序数据获得不同样品间差异表达的RNA列表,并对其进行生物学注释。
  \item 重测序项目经验:动植物相关重测序流程。
  \item 熟悉Python和R在生物信息上相关应用,如BioPython和Bioconductor,另外也使用perl、bash等作为日常语言。
  \item 在公司为客户(多是科研院校的研究人员)举办的RNA生物信息培训班上,连续两次担任R语言讲师,获得好评。
\end{itemize}


% \begin{onehalfspacing}
% \end{onehalfspacing}

% \datedsubsection{\textbf{DID-ACTE} 荷兰莱顿}{2015年3月 - 2015年6月}
% \role{本科毕业设计}{LIACS 交换生}
% 利用结巴分词对中国古文进行分词与词性标注,用已有领域知识训练形成 classifier 并对结果进行调优
% \begin{onehalfspacing}
% \begin{itemize}
%   \item 利用结巴分词对中国古文进行分词与词性标注
%   \item 利用已有领域知识训练形成 classifier, 并用分词结果进行测试反馈
%   \item 尝试不同规则,对 classifier 进行调优
% \end{itemize}
% \end{onehalfspacing}

\section{个人优势}
% increase linespacing [parsep=0.5ex]
\begin{itemize}[parsep=0.2ex]
  \item 科研背景,有文章发表经验,独立完成英文版初稿\cite{Zhao2015}。
  \item 编程语言能力突出,技术栈适应性广:8年Python经验,自学编程的第一门语言,对生物信息、爬虫、数据分析、深度学习、web开发等多种应用领域都有经验,对生态圈熟悉。对于调试、规范、测试、asyncIO等都有心得。R语言认真学过,了解R的LISP本质。Java/Groovy、Go、Node.js等语言都可以进行编程。
  \item DevOp能力强,Ansible熟练,Docker熟练,擅长流程部署与自动化。
  \item 对于开源软件生态有热情,乐于学习新的知识,勇于尝试新技术:Jsonnet、Nextflow、Ansible、Docker等技术的掌握主要来源于自主实践。
  \item 责任心强。
  \item 学习能力强,阅读英文文档是习惯。
  \item 乐于分享:个人认为技术分享是提高团队凝聚力的一项重要方式,而在帮助他人的过程中同时有利于加深对技术的理解。
  \item 合作精神:喜欢有效沟通+积极协作。

%   \item LeetCodeOJ Solutions, \textit{https://github.com/hijiangtao/LeetCodeOJ}
%   \item 第三届中国软件杯大学生软件设计大赛\textbf{全国一等奖}( \textit{http://www.cnsoftbei.com/} ),2014 年8月
%   \item 中国机器人大赛创意设计大赛\textbf{全国特等奖}( \textit{http://www.rcccaa.org/} ),2013年8月
% %   \item 中国机器人大赛暨Robocup公开赛(武术擂台赛)全国一等奖,2013年10月
%   \item 第11届北京理工大学“世纪杯”竞赛学生课外科技作品竞赛\textbf{特等奖},2013年8月
%   \item VIS Components for security system, \textit{https://hijiangtao.github.io/ss-vis-component/}
%   \item 个人博客:\textit{https://hijiangtao.github.io/},更多作品见 \textit{https://github.com/hijiangtao}
%   \item 电视节目"爸爸去哪儿"可视化分析展示, \textit{https://hijiangtao.github.io/variety-show-hot-spot-vis/}
\end{itemize}

\bibliographystyle{plain}
\renewcommand\refname{发表文章}
\bibliography{a}

% \begin{itemize}
%   \item Regulation of \textit{OsmiR156h} through Alternative Polyadenylation Improves Grain Yield in Rice. \\
%   Published: May 8, 2015 \textbf{Plos One} url: \textit{https://doi.org/10.1371/journal.pone.0126154}
% \end{itemize}

% \section{\faHeartO\ 项目/作品摘要}
% \section{项目/作品摘要}
% \datedline{\textit{An Integrated Version of Security Monitor Vis System}, https://hijiangtao.github.io/ss-vis-component/ }{}
% \datedline{\textit{Dark-Tech}, https://github.com/hijiangtao/dark-tech/ }{}
% \datedline{\textit{融合社交网络数据挖掘的电视节目可视化分析系统}, https://hijiangtao.github.io/variety-show-hot-spot-vis/}{}
% \datedline{\textit{LeetCodeOJ Solutions}, https://github.com/hijiangtao/LeetCodeOJ}{}
% \datedline{\textit{Info-Vis}, https://github.com/ISCAS-VIS/infovis-ucas}{}


% \section{\faInfo\ 社会实践/其他}
% \section{社区参与/实践其他}
% % increase linespacing [parsep=0.5ex]
% \begin{itemize}[parsep=0.2ex]
%   \item 乐于参与开源社区讨论,\textbf{参与翻译 Vue.js, webpack, WebAssembly, Babel 文档,印记中文成员}
%   \item 中国科学院大学2016秋季学期可视化与可视分析课程助教,\textit{http://vis.ios.ac.cn/infovis-ucas/}
%   \item 未来论坛学生会成员、北理社联新闻信息中心主任、北理工软件学院学生会宣传部副部长(2012-2016)
%   \item 2013-2015 北京市共青团“温暖衣冬”志愿者,第九届园博会志愿者,2014 FLL机器人世锦赛志愿者
% \end{itemize}

%% Reference
%\newpage
%\bibliographystyle{IEEETran}
%\bibliography{mycite}
\end{document}
